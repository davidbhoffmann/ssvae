% We did it. This paper rocks and you are lucky to have read it (i.e. brief recap of the entire paper). Also, we’ll do all these other amazing things in the future. 

In this seminar thesis, we provided a comprehensive review and empirical evaluation of the \ac{SSVAE} framework proposed by \cite{narayanaswamy_learning_2017}. By unifying probabilistic graphical models with deep generative networks, this framework offers a principled method for injecting domain knowledge through partial supervision into the representation learning process. This approach addresses the fundamental limitations of purely unsupervised disentanglement, which \cite{locatello_challenging_2018} later proved to be theoretically impossible without such inductive biases.

Through our systematic literature review, we identified critiques such as the semantic conflation problem \citep{joy_learning_2021}, which highlights that rigid partitioning of latent spaces can inadvertently starve continuous variables of semantic content. Furthermore, the field has largely evolved from the semi-supervised paradigm toward weak supervision \citep{locatello_weakly_2020} and causal disentanglement \citep{zhang_causal_2020} to overcome the reliance on expensive explicit labels.

Our own empirical experiments extended the original analysis of \cite{narayanaswamy_learning_2017} by investigating the model's sensitivity to supervision quality and intensity. We demonstrated that the \ac{SSVAE} is remarkably robust to label corruption, maintaining high classification accuracy even when 20\% of the supervision signal is randomised. 

Ultimately, while newer methods have refined the mechanisms for disentanglement, the \ac{SSVAE} remains a foundational framework. It successfully demonstrates that combining structured probabilistic modelling with the flexibility of deep neural networks is a viable path for learning interpretable, data-efficient representations.